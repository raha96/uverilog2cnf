\documentclass{article}

\usepackage{listings}
\usepackage{xcolor}
\usepackage{courier}
\usepackage{graphicx}
\usepackage{float}
\usepackage[hidelinks]{hyperref}

\definecolor{lightgray}{gray}{0.95}

\lstset{
	basicstyle=\ttfamily,
	backgroundcolor=\color{lightgray},
	numbers=left,
	tabsize=3,
	frame=tblr,
    breaklines=true, 
    postbreak=\mbox{\textcolor{red}{$\hookrightarrow$}\space}
}

\title{uverilog2cnf\\Upgraded Verilog2CNF}
\date{July 2021}
\author{Mohammad (Raha) Moradi Shahmiri\\raham9619@gmail.com}



\begin{document}
	\pagenumbering{gobble}
	\maketitle
	%\newpage
	%\tableofcontents
	\newpage

	\pagenumbering{arabic}

	\section{Introduction}
	uverilog2cnf, a.k.a uv2cnf, is meant as a tool for converting verilog gate-level descriptions to 
    cnf output suitable for debugging. Currently it supports conversion of a gate-level description to a standard 
    cnf file, alongside variable-to-number map. The library should be supplied in \textit{cnflib} format, which is a simple 
    custom format for storing CNF description of a cell library. Also a custom solution may be provided in \textit{sol} format 
    as well, so that the generated output includes the solution clauses. 

    \subsection{TODO}
    Support for standard \textit{liberty} format used for standard cell description was meant to be available in the first 
    version, but, as new nets might be created during conversion of liberty cells to predefined clauses, this feature was abandoned 
    for the current version, because new net names would greatly alter the debug process. Also as this tool is a work in progress, 
    support for the Verilog format is pretty limited. Support for assign statements is meant to be added in next versions, in parallel 
    to the support for liberty function format, as both require similar processing and synthesis. 

    \subsection {Usage}
    \textit{uverilog2cnf <input.v> <library.cnflib> <cnfoutput-prefix> <mapoutput-prefix> [solution.sol] [cardinality.card]}

    \textit{input.v} denotes the input gate-level verilog file.

    \textit{library.cnflib} denotes the cell library. The format shall be described further on.

    \textit{cnfoutput-prefix}  and \textit{mapoutput-prefix} are the prefixes used for generating output files. 
    For a module named \textit{-sample-module}, the output files would be 

    \lstinline{cnfoutput-prefix-samplemodule.cnf}

    \lstinline{mapoutput-prefix-samplemodule.map}

    \textit{solution.sol} (optional) refers to a solution file in sol format. The format shall be described further on.

    \textit{cardinality.card} (optional) refers to a cardinality condition description. The format shall be described further on.

    Sample files needed for a simple proof of work are available in the \textit{sample} folder and can be run by executing 
    \textit{run.sh}.

    \subsection{cnflib format}
    cnflib format is based on the unweighted cnf format. The file should begin with the number of cells to be described. 
    For each cell, its name is given. Then, like cnf format, number of variables (e.g. cell's ports) and clauses are listed.
    In the next line, the name of each port is givern, which shall be used for port mapping, as well as improving readbillity of 
    the generated variable map. Then each clause is listed as in cnf format. 

    \lstinputlisting[caption={Sample cnflib},label={lst:cnflib} ]{samples/sample.cnflib}

    \subsection{sol format}
    This format is designed to show arbitrary number of solutions, with each solution being an assignment to an arbitrary number 
    of variables. The file begins with the number of solutions. 
    Each solution begins with a number denoting the number of assignments in that solution. Then each assignment is shown as the 
    variable name followed by \textit{0} or \textit{1} as its value. 
    This feature is intended to simplify solution clause insertion, by automaticaully applying the generated variable map to 
    variable names and generating the clauses. 

    \lstinputlisting[caption={Sample sol file},label={lst:sol} ]{samples/sample.sol}

    \subsection{card format}
    Like \textit{sol}, \textit{card} is meant to simplify insertion of cardinality clauses by automatically applying the variable map. 
    Currently only 1-hot cardinality is implemented. 

    \textit{num-of-variables number-of-1s-allowed [var1] [var2] ...}

    \lstinputlisting[caption={Sample card file},label={lst:card} ]{samples/sample.card}

\end{document}
